\documentclass[a4paper,12pt,openany]{article}
\usepackage[utf8]{inputenc}
\usepackage[italian]{babel}
\usepackage{amsmath}
\usepackage{amssymb}
\usepackage{stmaryrd}
\usepackage{mathptmx}  % Font
\usepackage{relsize}   % Displaystyle for math symbols
\usepackage[linktocpage=true,hyperfootnotes=false,colorlinks=true,linkcolor=orange]{hyperref}  % Link nell'indice
\usepackage{microtype} % migliora l'uso di spazi e rientri
\usepackage{bbding}    % New symbols for itemize
\usepackage{setspace}  % Permette di personalizzare l'interlinea
\usepackage{graphicx}  % Permette di includere figure
\usepackage{wrapfig}
\usepackage{booktabs}  % Creazione tabelle
\usepackage{caption}   % Permette la personalizzazione delle didascalie e la loro aggiunta in tabelle
\usepackage[wide]{sidecap}  % Permette di mettere didascalie laterali a figure e tabelle
\usepackage{subfig}    % Crea sottofigure o sottotabelle
\usepackage[]{xcolor}  % Add colors
\usepackage{enumitem}
\usepackage{tikz}				% includo il pacchetto per disegnare la FSM
\usetikzlibrary{ arrows,automata}
\usepackage{listings}

\lstset{columns=fullflexible,
    mathescape=true,
    literate=
        {=}{$\leftarrow{}$}{1}
        {==}{$={}$}{1}
        {!=}{$\neq$}{1}
        {<=}{$\leq$}{1}
        {>=}{$\geq$}{1}
        {^2}{$^2$}{1},
    morekeywords={if,then,else,return}
}

\renewcommand\labelitemi{\footnotesize\SnowflakeChevron}

\newcommand{\TL}{T\!L}
\newcommand{\VB}{V\!B}
\newcommand{\RD}{R\!D}

\title{Analisi Statica}
\date{}

\begin{document}
    \maketitle
    
    \section*{Introduzione}
    L'analisi statica mira all'estrazione di proprietà sintattiche o semantiche valide \emph{per ogni esecuzione} di un dato programma P. Distinguiamo principalmente due tipi di analisi: \emph{distributive} ed \emph{non-distributive}. Le prime usualmente studiano \textbf{come} avanza il flusso di esecuzione, mentre le altre si concentrano su \textbf{cosa} viene calcolato dal programma.\\[0.5em]
    Tali analisi usano come strumento principe il \emph{CFG} (\emph{Control Flow Graph}), ovvero un grafo che rappresenta il flusso di controllo di un programma o di una procedura. È rappresentato mediante un grafo orientato con possibili cicli, avente un unico punto di ingresso e un unico punto di uscita, i cui nodi sono \emph{basic block} mentre gli archi sono le possibili transizioni tra blocchi.\\
    Un basic block è una sequenza di istruzioni del programma priva di costrutti di controllo (tutte le istruzioni che lo compongono sono eseguite sequenzialmente, senza salti o cicli). Per costruirlo quindi si crea un blocco ogni volta che c'è un salto.\\[3em]
    
    \textbf{QUALCOSA SULL'ASTRAZIONE}

    \newpage    
    \textbf{Prima distinzione}:
    \begin{itemize}
        \item Forward
        \begin{align*}
            \mbox{FAin}(n) = 
            \begin{cases}
                \emptyset & n=entry\\
                \bigoplus\limits_{m\in \mbox{Pred}(n)} \mbox{FAout}(m) & \mbox{otherwise}
            \end{cases}&\\[1em]
            \mbox{FAout}(m) = \tau(\mbox{FAin}(m))&\\
            \tau(\mbox{FAin}(m)) = \mbox{gen}(m) \cup (\mbox{FAout}(m) &\setminus \mbox{kill}(m))
        \end{align*}
        \item Backward
        \begin{align*}
            \mbox{BAout}(n) =
            \begin{cases}
                \emptyset & n=exit\\
                \bigoplus\limits_{m\in \mbox{Succ}(n)} \mbox{BAin}(m) & \mbox{otherwise}
            \end{cases}&\\[1em]
            \mbox{BAin}(m) = \tau(\mbox{BAout}(m))&\\
            \tau(\mbox{BAout}(m)) = \mbox{gen}(m) \cup (\mbox{BAin}(m) &\setminus \mbox{kill}(m))
        \end{align*}
    \end{itemize}

    \textbf{Seconda distinzione}:
    \begin{itemize}
        \item Possible analysis: $\bigoplus = \bigcup$
        \item Definite analysis: $\bigoplus = \bigcap$
    \end{itemize}
    
    \clearpage\section*{Available Expressions}
    \textbf{Forward \& Definite}\\[1em]
    Un espressione del tipo $x\leftarrow e$ si dice \emph{available} (disponibile) in un dato punto $P$ di programma se:
    \begin{itemize}[noitemsep, topsep=0.3em]
        \item È definita in un blocco $B$ precedente a $P$
        \item $x$ non è ridefinita tra $B$ e $P$
        \item Le variabili di $e$ non sono ridefinite tra $B$ e $P$
    \end{itemize}
    Per ottimizzare il codice si può sostituire il calcolo di un'espressione disponibile con la deferenziazione della variabile in cui è stata assegnata.\\[1em]
    
    \noindent\textbf{Proprietà dei basic blocks}
    \begin{align*}
        \mbox{AvailIn}(n) &=
        \begin{cases}
            \emptyset & n = entry\\
            \bigcap\limits_{m\in \mbox{Pred}(n)} \mbox{AvailOut}(m)
        \end{cases}\\
        \mbox{AvailOut}(m) &= \mbox{Gen}(m) \cup (\mbox{AvailIn}(m) \setminus \mbox{Kill}(m))\\
        \mbox{Gen}_A(n) &= \left\{
            x \leftarrow e \;\Bigg\vert
            \begin{array}{cc}
                x \leftarrow e \in n\\
                \land\\
                x \notin \mbox{Var}(e)
            \end{array}
        \right\}\\
        \mbox{Kill}_A(n) &= \left\{
            x \leftarrow e \;\big\vert\, \exists y \leftarrow e' \in n.\; \big( y \in \mbox{Var}(e) \,\lor\, y=x \big)
        \right\}
    \end{align*}
    
    \noindent\textbf{Semantica}
    \begin{align*}
        \mbox{Dominio: }& \mathbb{P}(\mbox{Ass})\\
        \llbracket ; \rrbracket^\#A &= A\\
        \llbracket \mbox{zero}(e) \rrbracket^\#A &= A\\
        \llbracket \mbox{non-zero}(e) \rrbracket^\#A &= A\\
        \llbracket M[e_1]\leftarrow e_2 \rrbracket^\#A &= A\\
        \llbracket x \leftarrow e \rrbracket^\#A &=
        \begin{cases}
            A \setminus \mbox{Occ}(x) \cup \{x\leftarrow e\} & x\notin \mbox{Var}(e)\\
            A \setminus \mbox{Occ}(x) & \mbox{otherwise}
        \end{cases}\\
        \llbracket x\leftarrow M[e] \rrbracket^\#A &= A\setminus \mbox{Occ}(x)\\
        \mbox{{\footnotesize where} Occ}(x) &=
        \left\{
            y\leftarrow e \;\Bigg\vert
            \begin{array}{cc}
                x \in \mbox{Var}(e)\\
                \lor\\
                y=x
            \end{array}
        \right\}
    \end{align*}
    
    
    \clearpage\section*{Liveness}
    \textbf{Backward \& Possible}\\[1em]
    Una variabile è viva in ogni punto di programma tra la sua definizione e il suo ultimo uso. Per calcolarle partiamo dal fondo e saliamo fino ad un uso della variabile; questa variabile resta viva finchè non troviamo la sua definizione, sopra la quale risulta non viva. Se una avariabile è viva vuol dire che il suo valore può essere letto. È un'analisi di tipo possible, poichè è sufficiente che sia viva in un ramo per renderla viva in quel punto di programma.
    
    Questa analisi risulta utile per cercare codice morto, variabili non inizializzate o registri liberi.\\[1em]
    
    \noindent\textbf{Proprietà dei basic blocks}
    
    \begin{align*}
    \mbox{LiveOut}(n) &=
    \begin{cases}
        \emptyset & n = exit\\
        \bigcup\limits_{m\in \mbox{Succ}(n)} \mbox{LiveIn}(m)
    \end{cases}\\
    \mbox{LiveIn}(m) &= \mbox{Gen}(m) \cup (\mbox{LiveOut}(m) \setminus \mbox{Kill}(m))\\
    \mbox{Gen}_{L}(n) &= \left\{
    x \;\big\vert\; \exists e \in n.\; x\in\mbox{Var}(e)
    \right\}\\
    \mbox{Kill}_{L}(n) &= \left\{
    x \;\big\vert\; \exists x\leftarrow e \in n
    \right\}
    \end{align*}
    
    \textbf{Semantica}
    \begin{align*}
    \mbox{Dominio: }& \mathbb{P}(\mbox{Var})\\
    \llbracket ; \rrbracket^\#L &= L\\
    \llbracket \mbox{zero}(e) \rrbracket^\#L &= L \cup \mbox{Var}(e)\\
    \llbracket \mbox{non-zero}(e) \rrbracket^\#L &= L \cup \mbox{Var}(e)\\
    \llbracket M[e_1]\leftarrow e_2 \rrbracket^\#L &= L \cup \mbox{Var}(e_1) \cup \mbox{Var}(e_2)\\
    \llbracket x \leftarrow e \rrbracket^\#L &= (L\setminus \{x\}) \cup \mbox{Var}(e)\\
    \llbracket x\leftarrow M[e] \rrbracket^\#L &= (L\setminus \{x\}) \cup \mbox{Var}(e)\\
    \end{align*}
    
    
    \clearpage\section*{True Liveness}
    \textbf{Backward \& Possible}
    \begin{align*}
    \mbox{TLiveOut}(n) &=
    \begin{cases}
    \emptyset & n = exit\\
    \bigcup\limits_{m\in \mbox{Succ}(n)} \mbox{TLiveIn}(m)
    \end{cases}\\
    \mbox{TLiveIn}(m) &= \mbox{Gen}(m) \cup (\mbox{TLiveOut}(m) \setminus \mbox{Kill}(m))\\
    \mbox{Gen}_{\TL}(n) &= \left\{
    x \;\big\vert\; x \in \TL \land \exists e \in n.\; x\in\mbox{Var}(e)
    \right\}\\
    \mbox{Kill}_{\TL}(n) &= \left\{
    x \;\big\vert\; \exists x\leftarrow e \in n
    \right\}
    \end{align*}
    
    \textbf{Semantica}
    \begin{align*}
    \mbox{Dominio: }& \mathbb{P}(\mbox{Var})\\
    \llbracket ; \rrbracket^\#\TL &= \TL\\
    \llbracket \mbox{zero}(e) \rrbracket^\#\TL &= \TL \cup \mbox{Var}(e)\\
    \llbracket \mbox{non-zero}(e) \rrbracket^\#\TL &= \TL \cup \mbox{Var}(e)\\
    \llbracket M[e_1]\leftarrow e_2 \rrbracket^\#\TL &= \TL \cup \mbox{Var}(e_1) \cup \mbox{Var}(e_2)\\
    \llbracket x \leftarrow e \rrbracket^\#\TL &=
    \begin{cases}
        (\TL\setminus \{x\}) \cup \mbox{Var}(e) & x\in\TL\\
        (\TL\setminus \{x\}) & \mbox{otherwise}
    \end{cases}\\
    \llbracket x\leftarrow M[e] \rrbracket^\#\TL &=
    \begin{cases}
        (\TL\setminus \{x\}) \cup \mbox{Var}(e) & x\in\TL\\
        (\TL\setminus \{x\}) & \mbox{otherwise}
    \end{cases}\\
    \end{align*}


    \clearpage\section*{Very Busy Expressions}
    \textbf{Backward \& Definite}
    \begin{align*}
    \mbox{\VB Out}(n) &=
    \begin{cases}
    \emptyset & n = exit\\
    \bigcap\limits_{m\in \mbox{Succ}(n)} \mbox{\VB In}(m)
    \end{cases}\\
    \mbox{\VB In}(m) &= \mbox{Gen}(m) \cup (\mbox{\VB Out}(m) \setminus \mbox{Kill}(m))\\
    \mbox{Gen}_{\VB}(n) &= \left\{
    x\leftarrow e \in n\;\big\vert\; x \notin\mbox{Var}(e)
    \right\}\\
    \mbox{Kill}_{\VB}(n) &= \left\{
        x\leftarrow e \;\Big\vert\; \exists y\leftarrow e'\in n.\,
        \begin{array}{cc}
            x\in \mbox{Var}(e')\quad\lor\\
            y\in \mbox{Var}(e)\;\quad\lor\\
            x = y
        \end{array}
    \right\}
    \end{align*}
    
    \textbf{Semantica}
    \begin{align*}
    \mbox{Dominio: }& \mathbb{P}(\mbox{Ass})\\
    \llbracket ; \rrbracket^\#\VB &= \VB\\
    \llbracket \mbox{zero}(e) \rrbracket^\#\VB &= \VB \setminus \mbox{Ass}(e)\\
    \llbracket \mbox{non-zero}(e) \rrbracket^\#\VB &= \VB \setminus \mbox{Ass}(e)\\
    \llbracket M[e_1]\leftarrow e_2 \rrbracket^\#\VB &= \VB \setminus (\mbox{Ass}(e_1) \cup \mbox{Ass}(e_2))\\
    \llbracket x \leftarrow e \rrbracket^\#\VB &= 
    \begin{cases}
    \VB \setminus (\mbox{Occ}(x) \cup \mbox{Ass}(e)) \cup \{x\leftarrow e\}& \mbox{if } x \notin \mbox{Var}(e)\\
    \VB \setminus (\mbox{Occ}(x) \cup \mbox{Ass}(e)) & \mbox{otherwise}
    \end{cases}\\
    \llbracket x\leftarrow M[e] \rrbracket^\#\VB &= \VB \setminus (\mbox{Occ}(x) \cup \mbox{Ass}(e))\\[1em]
    \mbox{{\footnotesize where} Occ}(x) &=
    \left\{
        y\leftarrow e \;\Bigg\vert
        \begin{array}{cc}
            x \in \mbox{Var}(e)\\
            \lor\\
            y=x
        \end{array}
    \right\}\\
    \mbox{{\footnotesize and} Ass}(e) &=
    \left\{
    y\leftarrow e' \in Ass \;\big\vert\;
    y \in \mbox{Var}(e) \,\land\, e \neq M[e]
    \right\}\\
    \end{align*}


    \clearpage\section*{Reaching Definitions}
    \textbf{Forward \& Possible}\\[1em]
    Le reaching definitions ci dicono, per ogni punto di programma, le variabili che sono disponibili e in che punto di programma sono state definite. Per ogni punto quindi si ha un insieme di coppie. A seconda del cammino percorso le definizioni disponibili potrebbero essere diverse, per questo ad una variabile possono corrispondere più punti di programma, rendendo così l'analisi di tipo possible.
    \\[1em]
    
    \noindent\textbf{Proprietà dei basic blocks}
    \begin{align*}
        \mbox{\RD In}(n) &=
        \begin{cases}
            \emptyset & n = entry\\
            \bigcup\limits_{m\in \mbox{Pred}(n)} \mbox{\RD Out}(m)
        \end{cases}\\
        \mbox{\RD Out}(m) &= \mbox{Gen}(m) \cup (\mbox{\RD In}(m) \setminus \mbox{Kill}(m))\\
        \mbox{Gen}_{\RD}(n) &= \left\{
            (x,n) \;\big\vert\; \exists x\leftarrow e \in n
        \right\}\\
        \mbox{Kill}_{\RD}(n) &= \left\{
            (x,n') \;\big\vert\; \exists x\leftarrow e \in n \land \exists (x,n')\in\RD(n)
        \right\}
    \end{align*}
    
    \textbf{Semantica}
    \begin{align*}
        \mbox{Dominio: }& \mathbb{P}(\mbox{Var $\times$ ProgPoints})\\
        \llbracket ; \rrbracket^\#\RD &= \RD\\
        \llbracket \mbox{zero}(e) \rrbracket^\#\RD &= \RD\\
        \llbracket \mbox{non-zero}(e) \rrbracket^\#\RD &= \RD\\
        \llbracket M[e_1]\leftarrow e_2 \rrbracket^\#\RD &= \RD\\
        \llbracket x \leftarrow e \rrbracket^\#\RD &= (\RD\setminus \mbox{Def}(x)) \cup
        \left\{
            (x, u) \;\big\vert\, K = (u, x\leftarrow e, v)
        \right\}\\
        \llbracket x\leftarrow M[e] \rrbracket^\#\RD &= (\RD\setminus \mbox{Def}(x)) \cup
        \left\{
            (x, u) \;\big\vert\, K = (u, x\leftarrow e, v)
        \right\}\\[1em]
        \mbox{{\footnotesize where} Def}(x) &=
        \left\{
            (x, n) \;\big\vert\, n \mbox{ punto di programma}
        \right\}\\
        \mbox{\footnotesize and } K &= 
        \begin{array}{ll}
            \mbox{Arco che stiamo analizzando, rappresentato da una}\\
            \mbox{tupla di: [nodo di partenza, istruzione, nodo di arrivo]}
        \end{array}
    \end{align*}


    \clearpage\section*{Copy Propagation}
    \textbf{Forward \& Definite}\\[1em]
    L'analisi di copy propagation ci dice, per ogni punto di programma, l'insieme di coppie di variabili che contengono lo stesso valore. È un'analisi di tipo forward e definite, che può essere utilizzata nelle ottimizzazione per evitare copie inutili. L'unica istruzione che genera una nuova coppia è $x\leftarrow y$, mentre ogni altra modifica di una di queste variabili uccide tutte le coppie in cui è presente.\\[1em]
    \noindent\textbf{Proprietà dei basic blocks}
    \begin{align*}
        \mbox{CopyIn}(n) &=
        \begin{cases}
        \emptyset & n = entry\\
        \bigcap\limits_{m\in \mbox{Pred}(n)} \mbox{CopyOut}(m)
        \end{cases}\\
        \mbox{CopyOut}(m) &= \mbox{Gen}(m) \cup (\mbox{CopyIn}(m) \setminus \mbox{Kill}(m))\\
        \mbox{Gen}_{C}(n) &= \left\{
        (x\; y) \;\big\vert\; \exists\, x\leftarrow y \in n
        \right\}\\
        \mbox{Kill}_{C}(n) &= \left\{
        (x\; y) \;\big\vert\; \exists\, x\leftarrow e \in n \,\lor\, \exists\, y\leftarrow e \in n
        \right\}
    \end{align*}
    
    \textbf{Semantica}
    \begin{align*}
        \mbox{Dominio: }& \mathbb{P}(\mbox{Var $\times$ Var})\\
        \llbracket ; \rrbracket^\#C &= C\\
        \llbracket \mbox{zero}(e) \rrbracket^\#C &= C\\
        \llbracket \mbox{non-zero}(e) \rrbracket^\#C &= C\\
        \llbracket M[e_1]\leftarrow e_2 \rrbracket^\#C &= C\\
        \llbracket x \leftarrow e \rrbracket^\#C &= C\setminus \mbox{Copie}(x)\quad e\notin Var\\
        \llbracket x \leftarrow M[e] \rrbracket^\#C &= C\setminus \mbox{Copie}(x)\\
        \llbracket x\leftarrow y \rrbracket^\#C &= (C\setminus \mbox{Copie}(x)) \cup
        \left\{
            (x\; z) \;\big\vert\; (z\; y) \in C
        \right\}\\[1em]
        \mbox{{\footnotesize where} Copie}(x) &=
        \left\{
        (x\; y) \;\big\vert\, (x\; y) \in C
        \right\}\\
    \end{align*}
    
    
    \clearpage\section*{Intervals}
    \textbf{Non-Distributive, Forward \& Possible}\\[1em]
    Questa analisi è solo semantica, in quanto ci dice cosa calcola un programma. Per ogni punto viene associato ad ogni variabile un intervallo che comprende i valori che questa può assumere. È perciò necessario definire nella semantica le operazioni tra intervalli per ogni possibile istruzione, incluse operazioni aritmetiche e di confronto.
    
    In alcuni casi però i cicli possono essere ripetuti molte volte, per cui si utilizza il \textbf{widening}: se dopo due iterazioni un estremo di un intervallo continua a crescere lo si approssima con l'infinito. Una volta raggiunto il punto fisso è possibile applicare il narrowing per restringere l'intervallo.\\[1em]

    \textbf{Semantica}
    \begin{align*}
    \mbox{Dominio: }& \mathbb{I}=\{[l,\,u]\;\big\vert\; l\in\mathbb{Z}\cup\{-\infty\},\; u\in\mathbb{Z}\cup\{+\infty\},\; l\leq u\}\\
    \llbracket ; \rrbracket^\#I &= I\\
    \llbracket \mbox{zero}(e) \rrbracket^\#I &= 
    \begin{cases}
        \bot & [0, 0] \not\sqsubseteq \llbracket e \rrbracket^\#I\\
        I' & otherwise
    \end{cases}\\
    \llbracket \mbox{non-zero}(e) \rrbracket^\#I &= 
    \begin{cases}
        \bot & [0, 0] = \llbracket e \rrbracket^\#I\\
        I' & otherwise
    \end{cases}\\
    \llbracket M[e_1]\leftarrow e_2 \rrbracket^\#I &= I\\
    \llbracket x \leftarrow e \rrbracket^\#I &= I\oplus \{x\mapsto\llbracket e \rrbracket^\#I\}\\
    \llbracket x \leftarrow M[e] \rrbracket^\#I &= I\oplus \{x\mapsto\top\}\\
    \mbox{\footnotesize where } I' &= (I \oplus I(x\mapsto I\{x\} \sqcap [-\infty,\; u]))\; \lor\\
    & \hspace{1.25em} (I \oplus I(x\mapsto I\{x\} \sqcap [l,\; +\infty]))\;\; \mbox{\textit{According to} } \llbracket e \rrbracket^\#I
    \end{align*}


    \clearpage\section*{Segni}
    \textbf{Non-Distributive, Forward \& Possible}\\[1em]
    Quella dei segni è un'analisi simile agli intervalli, ma semplificata, in cui le variabili possono assumere, nel dominio astratto, solo i valori positivo o negativo, con lo 0 incluso. Anche in questo caso dobbiamo definire le come le operazioni aritmetiche modificano il dominio.\\[1em]
    
    \textbf{Semantica}
    \begin{align*}
    \mbox{Dominio: }& \mathbb{S}=\{ \top, \mathbb{Z}_0^+, \mathbb{Z}^+, \mathbb{Z}_0^-, \mathbb{Z}^-, 0, \neq\!0, \bot \}\\
    \llbracket ; \rrbracket^\#S &= S\\
    \llbracket \mbox{zero}(e) \rrbracket^\#S &= 
    \begin{cases}
    \bot & [0, 0] \not\sqsubseteq \llbracket e \rrbracket^\#S\\
    S' & otherwise
    \end{cases}\\
    \llbracket \mbox{non-zero}(e) \rrbracket^\#S &= 
    \begin{cases}
    \bot & [0, 0] = \llbracket e \rrbracket^\#S\\
    S' & otherwise
    \end{cases}\\
    \llbracket M[e_1]\leftarrow e_2 \rrbracket^\#S &= S\\
    \llbracket x \leftarrow e \rrbracket^\#S &= S\oplus \{x\mapsto\llbracket e \rrbracket^\#S\}\\
    \llbracket x \leftarrow M[e] \rrbracket^\#S &= S\oplus \{x\mapsto\top\}\\
    \mbox{\footnotesize where } S' &= \mbox{\textit{The sign evaluation of} } \llbracket e \rrbracket^\#S
    \end{align*}


    \newpage
    \clearpage\section*{Esercizi}
    \subsection*{Available Expressions}
    \begin{tabular}{cc}
    \begin{minipage}{0.5\textwidth}
    \textbf{Codice}
    \begin{lstlisting}{}
        y = 1;
        v = x;
        while y != z
            t = 2y;
            if t <= z then
                v = v^2;
                y = 2y;
            else
                v = vx;
                y = y+1;
    \end{lstlisting}
    %\columnbreak
    \end{minipage}&
    
    \noindent
    \begin{minipage}{0.5\textwidth}
    Ass $= \{y\leftarrow1,\;v\leftarrow x,\;t\leftarrow2y\}$\\[0.5em]
    \begin{tabular}{c | l | l}
        & Gen & Kill\\
        \hline
        1  & $y\leftarrow1$ & $t\leftarrow2y$\\
        2  & $v\leftarrow x$ & $\emptyset$\\
        3  & $\emptyset$ & $\emptyset$\\
        4  & $t\leftarrow2y$ & $\emptyset$\\
        5  & $\emptyset$ & $\emptyset$\\
        6  & $\emptyset$ & $v\leftarrow x$\\
        7  & $\emptyset$ & $t\leftarrow2y,\; y\leftarrow1$\\
        8  & $\emptyset$ & $v\leftarrow x$\\
        9  & $\emptyset$ & $t\leftarrow2y,\; y\leftarrow1$\\
        10 & $\emptyset$ & $\emptyset$\\
    \end{tabular}
    \end{minipage}
    \end{tabular}

    \begin{minipage}{0.5\textwidth}
    \textbf{CFG - MOP}\\
    %\begin{wrapfigure}{l}{0.5\textwidth}
        \begin{tikzpicture}[>=stealth',auto,node distance=1cm]
        \small
        \node[label=right: $^0$] (0) {entry};
        \node[draw, below of=0, label=right: $^1$] (1) {$y\leftarrow1$};
        \node[draw, below of=1, label=right: $^2$] (2) {$v\leftarrow x$};
        \node[draw, below of=2, label=left: $^3$] (3) {$y\neq x$};
        \node[draw, below of=3, label=right: $^4$] (4) {$t\leftarrow2y$};
        \node[draw, below of=4, label=right: $^5$] (5) {$t\leq z$};
        \node[node distance = 2cm, draw, below left of=5, label=right: $^6$] (6) {$v\leftarrow v^2$};
        \node[draw, below of=6, label=right: $^7$] (7) {$y\leftarrow2y$};
        \node[node distance = 2cm, draw, below right of=5, label=right: $^8$] (8) {$v\leftarrow vx$};
        \node[draw, below of=8, label=right: $^9$] (9) {$y\leftarrow y+1$};
        \node[node distance = 3cm, right of=3, label=right: $^{10}$] (10) {exit};
        
        \path[->] (0) edge (1)
        (1) edge (2)
        (2) edge (3)
        (3) edge (4)
        (4) edge (5)
        (5) edge (6)
        (5) edge (8)
        (6) edge (7)
        (7) edge[bend left=65] (3)
        (8) edge (9)
        (9) edge[bend right=70] (3)
        (3) edge (10);
        
        \end{tikzpicture}
    %\end{wrapfigure}
    %\columnbreak
    \end{minipage}
    
    \begin{minipage}{0.5\textwidth}
    \begin{tabular}{c | l | l | l}
        AvailIn & $\emptyset$ & 1 & 2\\
        \hline
        1  & $\emptyset$ & $\emptyset$ & $\emptyset$\\
        2  & $T$ & $y\leftarrow1$ & $y\leftarrow1$\\
        3  & $T$ & $v\leftarrow x,\; y\leftarrow1$ & $\emptyset$\\
        4  & $T$ & $v\leftarrow x,\; y\leftarrow1$ & $\emptyset$\\
        5  & $T$ & $t\leftarrow2y,\; v\leftarrow x, y\leftarrow1$ & $t\leftarrow2y$\\
        6  & $T$ & $t\leftarrow2y,\; v\leftarrow x, y\leftarrow1$ & $t\leftarrow2y$\\
        7  & $T$ & $t\leftarrow2y,\; y\leftarrow1$ & $t\leftarrow2y$\\
        8  & $T$ & $t\leftarrow2y,\; v\leftarrow x, y\leftarrow1$ & $t\leftarrow2y$\\
        9  & $T$ & $t\leftarrow2y,\; y\leftarrow1$ & $t\leftarrow2y$\\
        10 & $T$ & $v\leftarrow x,\; y\leftarrow1$ & $\emptyset$\\
    \end{tabular}
    \end{minipage}

    
    \newpage
    \textbf{CFG - Semantics}\\
    \begin{tikzpicture}[>=stealth',auto,node distance=1.5cm]
    
    \node[draw, circle] (1) {$1$};
    \node[draw, circle, below of=1] (2) {$2$};
    \node[draw, circle, below of=2] (3) {$3$};
    \node[draw, circle, node distance=2cm, below left of=3] (4) {$4$};
    \node[draw, circle, below of=4] (5) {$5$};
    \node[draw, circle, node distance=2cm, below left of=5] (6) {$6$};
    \node[draw, circle, below of=6] (7) {$7$};
    \node[draw, circle, node distance=2cm, below right of=5] (8) {$8$};
    \node[draw, circle, below of=8] (9) {$9$};
    \node[draw, circle, node distance=2cm, xshift=5pt, below right of=3] (10) {$10$};


    
    \path[->] (1) edge node[right] {$y\leftarrow1$} (2)
              (2) edge node[right] {$v\leftarrow x$} (3)
              (3) edge node[right] {$y\neq z$} (4)
              (4) edge node[right] {$t\leftarrow2y$} (5)
              (5) edge node[right] {$t\leq z$} (6)
              (5) edge node[right] {$\neg\, t\leq z$} (8)
              (6) edge node[right] {$v\leftarrow v^2$} (7)
              (7) edge[bend left=70] node[right] {$y\leftarrow2y$} (3)
              (8) edge node[left] {$v\leftarrow vx$} (9)
              (9) edge[bend right=40] node[right] {$y\leftarrow y+1$} (3)
              (3) edge node[right] {$\neg\, y\neq z$} (10);
    
    \end{tikzpicture}



\end{document}